% !TEX TS-program = xelatex
% !TEX encoding = UTF-8 Unicode
% !Mode:: "TeX:UTF-8"

\documentclass{resume}
\usepackage{zh_CN-Adobefonts_external} % Simplified Chinese Support using external fonts (./fonts/zh_CN-Adobe/)
% \usepackage{NotoSansSC_external}
% \usepackage{NotoSerifCJKsc_external}
% \usepackage{zh_CN-Adobefonts_internal} % Simplified Chinese Support using system fonts
\usepackage{linespacing_fix} % disable extra space before next section
%\usepackage{cite}
%\usepackage[colorlinks,linkcolor=red]{hyperref}
%\hypersetup{colorlinks=true,linkcolor=red}
\usepackage{hyperref}


\begin{document}
\pagenumbering{gobble} % suppress displaying page number

\name{曾铭}

\basicInfo{
  \email{mingz1@foxmail.com} \textperiodcentered\ 
  \phone{(+86) 191-2171-1369} \textperiodcentered\ 
  \weixin{19121711369} \textperiodcentered\ 
  \year{26岁}
}

\section{教育背景}
\datedsubsection{\textbf{上海大学}, 上海}{2019 -- 2022}
\textit{硕士研究生}\ 人口学
\datedsubsection{\textbf{福建师范大学}, 福建, 福州}{2015 -- 2019}
\textit{学士}\ 城市管理

\section{工作经历}
\datedsubsection{\textbf{上海第一财经传媒有限公司}, 上海}{2022年8月 -- 至今}
\role{新一线城市研究所, 数据分析师}


\href{https://www.datayicai.com/home#/menu/}{知城数据平台-商业空间评估模块}
\begin{itemize}
  \item 持续维护和迭代商业空间评估功能模块中购物中心栏目,通过爬虫核查并更新购物中心的季度变化,更新存续情况。
  \item 构建购物中心的地理信息图层并持续维护购物中心的边界,支撑咨询项目的空间计算需求。
  \item 基于现有数据处理体系,用Python和n8n重构数据处理框架,持续优化数据管理流程。
  \item 执行数据提取、整合、分析更新、维护、开发对接等工作,
\end{itemize}

品牌爬虫项目库
\begin{itemize}
  \item 从零售、餐饮、出行等维度筛选细分领域品牌作为观测对象,抓取品牌门店信息。
  \item 搭建自托管 Crawlab 平台,维护并定时执行爬虫文件,监测门店数量变化。
\end{itemize}

项目支持
\begin{itemize}
  \item 基于空间算法计算识别汽车门店商圈,运用 shiny 等可视化工具搭建交互网页为业务同事提供支持。
  \item 对无印良品、龙湖和上海经济中心项目提供数据支持,与商业组协作对接客户需求。其中,龙湖项目中负责进行建模完成人口增量推算。
\end{itemize}

城市数据分析报告
\begin{itemize}
  \item {\href{https://mp.weixin.qq.com/s/SyOUXpGp4fwp5L7ENjbohw}{如何选择心仪大学?这里有一份高考志愿填报城市指南}}
    \begin{itemize}
      \item 根据高校文理科录取分数线、计划招生人数计算生源质量指数。
      \item 通过考察城市之间的生源质量指数差异,挖掘城市的未来潜力。
      \item 从指数结果分析考生对专业、校区所在城市的偏好。
    \end{itemize}
  \item {\href{https://mp.weixin.qq.com/s/E5gDIfpdBE4sHfa7M52udA}{都市圈内外的创新联系什么样?}}
    \begin{itemize}
      \item 基于城市之间的投资事件和企业分支数据,分析城市、都市圈之间的关联度。
    \end{itemize} 
\end{itemize}



% Reference Test
%\datedsubsection{\textbf{Paper Title\cite{zaharia2012resilient}}}{May. 2015}
%An xxx optimized for xxx\cite{verma2015large}
%\begin{itemize}
%  \item main contribution
%\end{itemize}

\section{技能}
\begin{itemize}
  \item \textbf{编程语言:} \small R, Python, SQL
  \item \textbf{工具:} \small MS Office, \LaTeX, PostgreSQL, Docker, Scrapy
  \item \textbf{语言:} \small 英语 - 熟练(英语六级:580)
\end{itemize}

\end{document}
